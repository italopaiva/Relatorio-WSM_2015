\subsection{Planejamento da aplicação da ontologia no ``Pé na Estrada''}

Atualmente, o “Pé na estrada” disponibiliza informações apenas sobre os locais dos
acidentes, e a quantidade de acidentes em cada rodovia. Esse tipo de informação seria mais
relevante se acompanhassem informações acerca dos tipos de acidentes, danos causados a
rodovia, tipo de veículo, envolvidos, entre outros dados que fossem relevantes para análise
estatística e para representação gráfica.

Dessa forma, com o modelo de dados estruturado semanticamente, esses dados seriam
encontrados e relacionados facilmente, possibilitando ao software disponibilizar essas
informações ao usuário, até mesmo de forma mais visual.

Atualmente, o software não possui nenhuma camada semântica presente, possuindo
apenas a camada visual do HTML, sendo necessária toda a estruturação das três principais
camadas semânticas: o XML, o RDFS e a OWL, sendo esta última necessária para o uso da
ontologia que será criada para representar os acidentes de trânsito.

O software ``Pé na Estrada'' foi desenvolvido na linguagem \textit{Ruby} com a utilização do \textit{framework Rails}. 
A partir disso, o sistema funciona com a arquitetura da Figura \ref{fig:arquitetura}

\begin{figure}[!htb]
 \centering
 \includegraphics[scale = 0.7]{figuras/arquiteturarails.png}
 \caption{Arquitetura do sistema}
 \label{fig:arquitetura}

\end{figure}