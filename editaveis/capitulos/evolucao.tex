\section{Evolução da \textit{Web}}
A internet como conhecemos hoje surgiu a partir de uma idealização, em meados dos
anos 60, de uma rede de comunicação militar alternativa capaz de resistir a um possível
conflito núclear mundial. Desenvolvida por um grupo de programadores e engenheiros
eletrônicos com o conceito de uma rede sem nenhum controle central onde cada
computador seria apenas um “elo” na transmissão das informações \cite{monteiro01}.

Desde então, a Web cresceu em um ritmo assustador. Porém, grande parte das páginas
disponíveis na Web hoje em dia ainda mantém as características da chamada Web 1.0.

\textbf{WEB 1.0}

Segundo Vicentim (\citeyear{vicentim13}), é a internet como ela surgiu. Sites de contéudo simples e
estáticos onde as informações eram interligadas através de diversos diretórios de links
relacionados para leitores humanos e não para máquinas e programas de computadores
(software). Os computadores eram utilizados meramente para exibir informação na tela. O
conteúdo da página “somente leitura” fazia do usuário um mero espectador.

\textbf{WEB 2.0}

O termo Web 2.0 (e consecutivamente, o Web 1.0) foi criado pelo especialista no setor
Tim O’Reilly, classificando essa nova forma de utilizar a internet como uma “web como
plataforma”.

“Surgiu por volta de 2004 e ao contrário da Web 1.0, é caracterizada pela interatividade e
participação do usuário final com a estrutura e conteúdo da página, fazendo do usuário um
contribuidor” \cite{lacerda12}.

Porém, com toda essa expansão, surgiu um grande problema: A dificuldade de lidar
com o excesso de informações inúteis e/ou erradas. A Internet atual é denominada Web
Sintática. As páginas da Web não contêm informações sobre si mesmas, ou seja, os9
computadores não identificam que tipo de conteúdo está disposto e a que assunto(s) a
página se refere, deixando o processo de interpretação a serviço dos seres humanos. O que
nos faz pensar: Por que os computadores não podem realizar esse trabalho?
A partir disso, vêm a motivação para uma Web do Futuro, a Web 3.0.

\textbf{WEB 3.0}

O termo Web 3.0 foi criado pelo jornalista John Markoff, do New York Times, baseado
na evolução do termo Web 2.0 criado por O’Really em 2004. Outras denominações desse
mesmo momento são “Web Semântica” ou “Web Inteligente”.

A ideia da Web Semântica surgiu em 2001, quando Tim Berners-Lee, James Hendler e
Ora Lassila publicaram um artigo na revista Scientific American, intitulado: “Web
Semântica: um novo formato de conteúdo para a Web que tem significado para
computadores vai iniciar uma revolução de novas possibilidades.”

A Web 3.0 pode ser definida como:

\begin{citacao} 
 Uma internet onde teremos toda informação de forma organizada para que não
somente os humanos possam entender, mas principalmente as máquinas, assim elas
podem nos ajudar respondendo pesquisas e perguntas com uma solução concreta,
personalizada e ideal. É uma internet cada vez mais próxima da inteligência
artificial. É um uso ainda mais inteligente do conhecimento e conteúdo já
disponibilizado online, com sites e aplicações mais inteligentes, experiência
personalizada e publicidade baseada nas pesquisas e no comportamento de cada
indivíduo \cite{vicentim13}.
\end{citacao}