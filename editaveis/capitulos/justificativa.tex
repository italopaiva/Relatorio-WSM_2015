\section[Justificativa]{Justificativa}

De acordo com o planejamento estratégico da PRF (\cite{prf13}), deseja-se aprimorar a
percepção de segurança dos usuários das rodovias federais e o registro de ocorrências.

Para o cumprimento desse planejamento estratégico, a disponibilização do software
“Pé na Estrada” para o uso dos cidadãos seria louvável. Mas para tal, é necessário
estruturar os dados referentes aos acidentes em rodovias federais, de modo a aperfeiçoar o
seu uso e analisar estatisticamente os acidentes ocorridos nas rodovias federais para
propor melhorias relacionadas ao trabalho realizado.

Esse trabalho foi iniciado em 2015 por alguns alunos da disciplina de Tópicos Especiais em Engenharia de Software da Universidade de Brasília com a criação de uma ontologia específica para a PRF, e o planejemento e modelagem de um modelo de dados semântico para o software "Pé na Estrada". Nesse semestre deseja-se finalizar o trabalho com a implementação desse modelo. 