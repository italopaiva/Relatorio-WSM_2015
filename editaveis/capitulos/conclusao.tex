\chapter{Conclusão}

A partir da realização deste trabalho foi possível conhecer sobre \textit{Web} Semântica e suas potencialidades. 
Bem como, conhecer sobre a aplicação de ontologias e seu impacto no domínio de aplicações \textit{Web}.

Embora as ontologias ainda não sejam amplamente utilizadas, já existem ontologias para vários domínios. No caso
desse trabalho foram pesquisadas ontologias de acidentes e foi possível perceber que embora não se adequassem
completamente ao contexto, seus conceitos puderam ser aproveitados.

A utilização da metodologia \textit{ontology} 101 permitiu a construção da ontologia em passos que resultaram na ontologia
de acidentes para uso no ``Pé na Estrada'' com a reutilização de ontologias existentes.

Com a ontologia criada podemos amplificar a potencialidade do \textit{software} ``Pé na Estrada'', permitindo, por exemplo,
pesquisas mais avançadas em dados mais interconectados.
Todas essas novas possibilidades gerariam novas funcionalidades
para o \textit{software} de extremo valor para o usuário.

É possível perceber que com o uso de ontologias e \textit{Web} Semântica possibilita o aumento da representatividade
da aplicação, visto que expande o domínio do ``Pé na Estrada'' agregando mais valor ao usuário, no que se refere a quantidade
de informações que o sistema fornece.

Foi possível ver que um banco de dados orientado a grafos pode ser mais intuitivo que um relacional, sendo até visualmente 
mais agradável que um banco de dados relacional. Embora o banco de dados orientado a grafos seja um pouco mais lento, os recursos
fornecidos por ele valem a pena para dados muito conectados.
