\section{Linguagens de ontologias}

A ideia inicial de Tim Bernes Lee para a web semântica é uma arquitetura baseada em
camadas, de forma que as camadas se sobreponham e se completem adicionando cada vez
mais semântica aos dados, mas mantendo a escalabilidade \cite{breitman05}. Dessa forma
é possível representar semanticamente uma página web da maneira mais simples para uma
máquina que não possua todo o arcabouço semântico necessário para visualizar todo o leque
de opções, da mesma forma que é possível outra máquina com capacidade de interpretar as
camadas semânticas superiores também visualizar da forma mais completa, semanticamente,
a página requisitada.

A primeira camada dessa arquitetura escalar da web é a Hypertext Markup Language
(HTML), que lida com o conteúdo da informação e sua apresentação para os seres humanos,
fornecendo pouca expressividade para os dados e suas estruturas \cite{breitman05}.
Subindo mais um degrau da arquitetura web, a forma mais simples de representar
semanticamente um documento é feita por meio da linguagem eXtensible Markup Language
(XML), que é uma linguagem de marcação flexível que permite a criação de tags pelo
usuário, o que possibilita criar uma representação semântica para cada pedaço da informação
de um documento e para as relações entre esses pedaços, relacionando-os por meio de
aninhamento das tags. O XML começou a se preocupar com a representação da estrutura do
documento apresentado \cite{breitman05}.

O Resource Description Framework (RDF) é uma linguagem declarativa que se baseia
no XML, fornecendo uma maneira padronizada de utilizá-lo. Uma das propostas do RDF é
fazer com que os recursos da web sejam legíveis e acessíveis por máquinas (BREITMAN,
2005). Permite descrever sentenças sobre propriedades e relacionamentos entre itens na web,
destacando três componentes para a sentença: recurso, propriedade e valor, que se
assemelham ao sujeito, predicado e objeto de uma oração qualquer da língua portuguesa,
respectivamente. O RDF conta com uma linguagem de suporte para descrever o vocabulário
que será utilizado pelo documento RDF, o RDF-Schema. O RDF veio para adicionar uma
camada mais semântica para a web, sobrepondo a camada já fornecida pelo XML.

O RDF e o RDF-Schema (RDFS, combinação entre as duas linguagens) representam a
base para a web semântica. Apesar de várias linguagens para ontologias importantes terem
sido propostas ao longo do tempo com base em extensões do RDFS (como SHOE, Oil,
DAML, DAML+Oil), será descrita apenas a Web Ontology Language (OWL), que acabou
representando um resultado final dos esforços das outras linguagens, e é hoje uma
recomendação do W3C no que tange a linguagens ontológicas \cite{breitman05}.

A OWL é uma linguagem que foi projetada para resolver problemas inerentes às aplicações
da Web Semântica, permitindo a construção de ontologias, explicitação de fatos sobre um
domínio específico e racionalizar sobre ontologias e fatos \cite{breitman05}. O objetivo
dessa linguagem é representar conceitos e seus respectivos relacionamentos como uma
ontologia, e possui três linguagens com diferentes níveis de expressividade: OWL Lite, OWL
DL e OWL Full.