\section{Resultados Esperados}

Espera-se com esse projeto, fornecer à PRF um modelo de dados mais representativo
para o domínio de ocorrências de acidentes, que provoque uma melhoria na
disponibilização dos dados abertos.

Ainda que represente um esforço considerável num período de 2 meses, com um custo total de 3088,50 R\$ a implantação da ontologia representa um aumento considerável na usabilidade da ferramenta, fornecendo dados semanticamente relacionados, possibilitando inclusive futuras implementações de inteligência artificial para filtragem dos dados. Dessa forma, considera-se que os benefícios para aplicação da ontologia superam o custo inerente para implantação e mudança da ferramenta. 

A equipe de projeto também acredita que o software “Pé na estrada” pode ser adotado como uma ferramenta de suporte a utilização das rodovias federais brasileiras para os usuários das mesmas.