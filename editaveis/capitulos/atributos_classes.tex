% Classe Acidente
\noindent
\emph{\textbf{Acidente}}

  A classe Acidente pode ser considerada como a classe central para a aplicação e para a ontologia,
  uma vez que todas as outras classes estão relacionadas a um acidente ocorrido.
  
    \begin{table*}[!h]
    \centering
    \begin{tabular}{p{0.15\linewidth}p{0.2\linewidth}p{0.5\linewidth}}
      \hline
      \textbf{Classe} & \textbf{Atributo} & \textbf{Descrição}\\
      \hline
	Acidente & data\_acidente & Data da ocorrência do acidente\\
		  & hora\_acidente & Horário da ocorrência do acidente\\ 
      \hline
    \end{tabular}
    \caption{Atributos da classe Acidente}
    \label{tab:attr_acidente}
    \end{table*}
    
% Classe Veículo
\noindent
\emph{\textbf{Veículo}}

  A classe Veículo representa os veículos envolvidos em um acidente.
  
    \begin{table*}[!h]
    \centering
    \begin{tabular}{p{0.15\linewidth}p{0.22\linewidth}p{0.5\linewidth}}
      \hline
      \textbf{Classe} & \textbf{Atributo} & \textbf{Descrição}\\
      \hline
	Veiculo & tipo\_veiculo & Tipo do veículo (Ex.: Hatch, sedan, caminhonete)\\
		& modelo\_veiculo & Modelo do veículo\\
		& ano\_veiculo & Ano do veículo\\		
		& km\_veiculo & Quilometragem do veículo no momento do acidente\\
		& defeitos\_veiculo & Defeitos encontrados no veículo no momento do acidente\\
		& quantidade\_pessoas & Quantidade de pessoas dentro do veículo no momento do acidente\\
      \hline
    \end{tabular}
    \caption{Atributos da classe Veículo}
    \label{tab:attr_veiculo}
    \end{table*}

\vfill
\pagebreak

% Classe Ocupante
\noindent
\emph{\textbf{Ocupante}}

  A classe Ocupante representa as pessoas que podem estar dentro de um veículo no momento do acidente.
  
    \begin{table*}[!h]
    \centering
    \begin{tabular}{p{0.15\linewidth}p{0.23\linewidth}p{0.5\linewidth}}
      \hline
      \textbf{Classe} & \textbf{Atributo} & \textbf{Descrição}\\
      \hline
	Ocupante & sexo & Modelo do veículo\\
		 & idade\_pessoa & Idade da pessoa\\
		 & usava\_cinto & Indica se o ocupante do veículo estava usando o cinto de segurança\\
		 & estado\_mental* & Indica o estado mental do motorista no momento do acidente\\
		 & sob\_efeito\_toxicos* & Indica se o motorista estava sob efeito de tóxicos no momento do acidente\\
		 & uf\_cnh\_motorista* & UF da CNH do motorista\\
      \hline
    \end{tabular}
    *Atributos específicos para instâncias com a relação 'motorista'.

    \caption{Atributos da classe Ocupante}
    \label{tab:attr_ocupante}
    \end{table*}
    
% Classe Local
\noindent
\emph{\textbf{Local}}

  A classe Local representa o local, dentro de uma rodovia, no qual um acidente ocorreu.
  
    \begin{table*}[!h]
    \centering
    \begin{tabular}{p{0.15\linewidth}p{0.23\linewidth}p{0.5\linewidth}}
      \hline
      \textbf{Classe} & \textbf{Atributo} & \textbf{Descrição}\\
      \hline
	Local & latitude & Latitude em que ocorreu o acidente\\
	      & longitude & Longitude em que ocorreu o acidente\\
	      & km\_rodovia & Indica em qual quilômetro da rodovia ocorreu o acidente\\
	      & ponto\_referencia & Indica um ponto de referência próximo ao local do acidente\\
	      & regiao\_urbana & Indica se a área do acidente é uma região urbana\\
	      & volume\_trafego & Indica o volume de tráfego na região\\
      \hline
    \end{tabular}
    \caption{Atributos da classe Local}
    \label{tab:attr_local}
    \end{table*}

\vfill
\pagebreak
    
% Classe Rodovia
\noindent
\emph{\textbf{Rodovia}}

  A classe Rodovia representa as rodovias federais brasileiras. No domínio da aplicação, todo acidente ocorre em uma rodovia.
  
    \begin{table*}[!h]
    \centering
    \begin{tabular}{p{0.15\linewidth}p{0.23\linewidth}p{0.5\linewidth}}
      \hline
      \textbf{Classe} & \textbf{Atributo} & \textbf{Descrição}\\
      \hline
	Rodovia & nome\_rodovia & Nome da rodovia (Ex.: BR-060)\\
		& extensao\_rodovia & Indica o tamanho da rodovia\\
		& estado\_rodovia & Identifica quais estados brasileiros que a rodovia percorre\\
		& posto\_prf & Identifica um posto da PRF ao longo da rodovia\\
      \hline
    \end{tabular}
    \caption{Atributos da classe Rodovia}
    \label{tab:attr_rodovia}
    \end{table*}
    
\vfill    