% Classe Acidente
\subsubsection*{\textbf{Acidente}}

  A classe Acidente pode ser considerada como a classe central para a aplicação e para a ontologia,
  uma vez que todas as outras classes estão relacionadas a um acidente ocorrido.
  
    \begin{table*}[!h]
    \centering
    \begin{tabular}{p{0.15\linewidth}p{0.2\linewidth}p{0.5\linewidth}}
      \hline
      \textbf{Classe} & \textbf{Atributo} & \textbf{Descrição}\\
      \hline
	Acidente & data\_acidente & Data da ocorrência do acidente\\
		 & hora\_acidente & Horário da ocorrência do acidente\\
		 & danos\_causados & Descrição dos danos causados pelo acidente\\
      \hline
    \end{tabular}
    \caption{Atributos da classe Acidente}
    \label{tab:attr_acidente}
    \end{table*}
    
% Classe Veículo
\subsubsection*{\textbf{Veículo}}

  A classe Veículo representa os veículos envolvidos em um acidente.
  
    \begin{table*}[!h]
    \centering
    \begin{tabular}{p{0.15\linewidth}p{0.22\linewidth}p{0.5\linewidth}}
      \hline
      \textbf{Classe} & \textbf{Atributo} & \textbf{Descrição}\\
      \hline
	Veiculo & tipo\_veiculo & Tipo do veículo (Ex.: Hatch, sedan, caminhonete)\\
		& modelo\_veiculo & Modelo do veículo\\
		& ano\_veiculo & Ano do veículo\\		
		& km\_veiculo & Quilometragem do veículo no momento do acidente\\
		& defeitos\_veiculo & Defeitos encontrados no veículo no momento do acidente\\
		& quantidade\_pessoas & Quantidade de pessoas dentro do veículo no momento do acidente\\
      \hline
    \end{tabular}
    \caption{Atributos da classe Veículo}
    \label{tab:attr_veiculo}
    \end{table*}

% \vfill
% \pagebreak

% Classe Ocupante
\subsubsection*{\textbf{Ocupante}}

  A classe Ocupante representa as pessoas que podem estar dentro de um veículo no momento do acidente.
  
    \begin{table*}[!h]
    \centering
    \begin{tabular}{p{0.15\linewidth}p{0.23\linewidth}p{0.5\linewidth}}
      \hline
      \textbf{Classe} & \textbf{Atributo} & \textbf{Descrição}\\
      \hline
	Ocupante & sexo & Modelo do veículo\\
		 & idade\_pessoa & Idade da pessoa\\
		 & usava\_cinto & Indica se o ocupante do veículo estava usando o cinto de segurança\\
		 & estado\_mental* & Indica o estado mental do motorista no momento do acidente\\
		 & sob\_efeito\_toxicos* & Indica se o motorista estava sob efeito de tóxicos no momento do acidente\\
		 & uf\_cnh\_motorista* & UF da CNH do motorista\\
      \hline
    \end{tabular}
    *Atributos específicos para instâncias com a relação 'motorista'.

    \caption{Atributos da classe Ocupante}
    \label{tab:attr_ocupante}
    \end{table*}
    
% Classe Local
\subsubsection*{\textbf{Local}}

  A classe Local representa o local, dentro de uma rodovia, no qual um acidente ocorreu.
  
    \begin{table*}[!h]
    \centering
    \begin{tabular}{p{0.15\linewidth}p{0.23\linewidth}p{0.5\linewidth}}
      \hline
      \textbf{Classe} & \textbf{Atributo} & \textbf{Descrição}\\
      \hline
	Local & latitude & Latitude em que ocorreu o acidente\\
	      & longitude & Longitude em que ocorreu o acidente\\
	      & km\_rodovia & Indica em qual quilômetro da rodovia ocorreu o acidente\\
	      & ponto\_referencia & Indica um ponto de referência próximo ao local do acidente\\
	      & regiao\_urbana & Indica se a área do acidente é uma região urbana\\
	      & volume\_trafego & Indica o volume de tráfego na região\\
      \hline
    \end{tabular}
    \caption{Atributos da classe Local}
    \label{tab:attr_local}
    \end{table*}

% \vfill
% \pagebreak
    
% Classe Rodovia
\subsubsection*{\textbf{Rodovia}}

  A classe Rodovia representa as rodovias federais brasileiras. No domínio da aplicação, todo acidente ocorre em uma rodovia.
  
    \begin{table*}[!h]
    \centering
    \begin{tabular}{p{0.15\linewidth}p{0.23\linewidth}p{0.5\linewidth}}
      \hline
      \textbf{Classe} & \textbf{Atributo} & \textbf{Descrição}\\
      \hline
	Rodovia & nome\_rodovia & Nome da rodovia (Ex.: BR-060)\\
		& extensao\_rodovia & Indica o tamanho da rodovia\\
		& estado\_rodovia & Identifica quais estados brasileiros que a rodovia percorre\\
		& posto\_prf & Identifica um posto da PRF ao longo da rodovia\\
      \hline
    \end{tabular}
    \caption{Atributos da classe Rodovia}
    \label{tab:attr_rodovia}
    \end{table*}

% Classe Estatística    
\subsubsection*{\textbf{Estatística}}

  A classe Estatística representa as estatísticas que pode se inferir sobre os acidentes ocorridos em uma rodovia.
  
    \begin{table*}[!h]
    \centering
    \begin{tabular}{p{0.15\linewidth}p{0.23\linewidth}p{0.5\linewidth}}
      \hline
      \textbf{Classe} & \textbf{Atributo} & \textbf{Descrição}\\
      \hline
	Estatística & nome\_estatistica & Nome da estatística\\
		    & tipo\_estatistica & Indica o tipo da estatística\\
		    & valor\_estatistica & Identifica o valor da estatística\\
		    & desvio\_padrao & Indica o desvio padrão da estatística calculada\\
		    & data\_calculo & Indica a data em que foi calculada aquela estatística\\
      \hline
    \end{tabular}
    \caption{Atributos da classe Estatística}
    \label{tab:attr_estatistica}
    \end{table*}    

% Classe Causa    
\subsubsection*{\textbf{Causa}}

  A classe Causa representa as possíveis causas de uma acidente. Para trabalhos futuros, pretende-se especializar esta classe 
  nas diferentes causas de acidentes existentes ou, se possível, utilizar uma ontologia de causas de acidentes.
  
    \begin{table*}[!h]
    \centering
    \begin{tabular}{p{0.15\linewidth}p{0.23\linewidth}p{0.5\linewidth}}
      \hline
      \textbf{Classe} & \textbf{Atributo} & \textbf{Descrição}\\
      \hline
	Causa & descricao\_causa & Descrição da causa do acidente\\
	      & tipo\_causa & Indica o tipo da causa de acidente (Ex.: Falha técnica, condição da pista, etc.)\\
	      & periculosidade & Identifica o grau de periculosidade da causa\\
      \hline
    \end{tabular}
    \caption{Atributos da classe Causa}
    \label{tab:attr_causa}
    \end{table*}      
    
% Classe TipoAcidente    
\subsubsection*{\textbf{TipoAcidente}}

  A classe TipoAcidente representa os tipos de acidentes que podem estar relacionados a determinado acidente.
  
    \begin{table*}[!h]
    \centering
    \begin{tabular}{p{0.15\linewidth}p{0.23\linewidth}p{0.5\linewidth}}
      \hline
      \textbf{Classe} & \textbf{Atributo} & \textbf{Descrição}\\
      \hline
	TipoAcidente & nome\_tipo & Indica o nome do tipo de acidente\\
		     & descricao\_tipo & Descrição do tipo de acidente\\
      \hline
    \end{tabular}
    \caption{Atributos da classe TipoAcidente}
    \label{tab:attr_tipo_acidente}
    \end{table*}
    
% Classe PostosPRF    
\subsubsection*{\textbf{PostoPRF}}

  A classe PostoPRF representa os os postos da PRF que podem estar localizados em uma rodovia.
  Os postos da PRF localizam-se ao longo das rodovias, o que os classificam como um tipo de local definido na
  classe Local. Logo, a classe PostoPRF possuem os mesmos atributos da classe Local, como latitude e longitude,
  e mais os seguintes:
  
    \begin{table*}[!h]
    \centering
    \begin{tabular}{p{0.15\linewidth}p{0.23\linewidth}p{0.5\linewidth}}
      \hline
      \textbf{Classe} & \textbf{Atributo} & \textbf{Descrição}\\
      \hline
	PostoPRF & nome\_posto & Indica o nome do posto da PRF (Ex.: 10º posto da Polícia Rodoviária Federal)\\
		 & encarregado\_posto & Indica o nome do chefe responsável pelo posto da PRF\\
		 & telefone & Indica o número de telefone do posto da PRF\\
		 & email & Indica o endereço de \textit{e-mail} do posto da PRF\\
		 & detalhes\_posto & Informa detalhes sobre o posto da PRF (Ex.: Há local para pouso de aeronaves)\\
      \hline
    \end{tabular}
    \caption{Atributos da classe PostoPRF}
    \label{tab:attr_postoprf}
    \end{table*}
    
% Classe AssistenciaTecnica    
\subsubsection*{\textbf{AssistenciaTecnica}}

  A classe AssistenciaTecnica representa entidades que possam oferecer algum serviço de assistência aos motoristas, localizadas 
  ao longo ou nas cidades mais próximas de uma rodovia.
  
    \begin{table*}[!h]
    \centering
    \begin{tabular}{p{0.15\linewidth}p{0.23\linewidth}p{0.5\linewidth}}
      \hline
      \textbf{Classe} & \textbf{Atributo} & \textbf{Descrição}\\
      \hline
	AssistenciaTecnica & nome\_assistencia & Indica o nome da unidade de assistência técnica\\
			   & tipo\_assistencia & Indica o tipo de assistência oferecida (Ex.: Borracharia)\\
			   & endereco & Indica o endereço da unidade de assistência técnica\\
			   & telefone & Indica o número de telefone da unidade de assistência técnica\\
      \hline
    \end{tabular}
    \caption{Atributos da classe PostoPRF}
    \label{tab:attr_postoprf}
    \end{table*}
    
% Classe Hospital    
\subsubsection*{\textbf{Hospital}}

  A classe Hospital representa as unidades hospitalares localizadas próxima a uma rodovia. Para trabalhos futuros, pretende-se
  utilizar uma ontologia para a definição de hospitais.
  
    \begin{table*}[!h]
    \centering
    \begin{tabular}{p{0.15\linewidth}p{0.23\linewidth}p{0.5\linewidth}}
      \hline
      \textbf{Classe} & \textbf{Atributo} & \textbf{Descrição}\\
      \hline
	Hospital & nome\_hospital & Indica o nome do hospital\\
		 & especialidade & Indica qual a especialidade do hospital\\
		 & endereco & Indica o endereço do hospital\\
		 & telefone & Indica o número de telefone do hospital\\
      \hline
    \end{tabular}
    \caption{Atributos da classe Hospital}
    \label{tab:attr_hospital}
    \end{table*}
    
\vfill