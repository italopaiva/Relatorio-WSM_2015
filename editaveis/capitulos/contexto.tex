\section[Contexto]{Contexto}

De acordo com o Sistema de Notificação de Mortalidade (SIM), somente em 2012,
48.812 pessoas morreram em acidentes nas rodovias brasileiras e entre as maiores causas
estão: imprudência, ingestão de álcool, curvas perigosas, pistas mal conservadas e mal
projetadas. Segundo a Polícia Rodoviária Federal (2013), 20 rodovias no Brasil causam
mais de 100 óbitos por ano.

Com a Lei de acesso à informação (LEI 12.527/2011) (BRASIL, 2011), que garante
aos cidadãos o acesso a informações acerca dos órgãos públicos integrantes da
administração direta, a Polícia Rodoviária Federal (PRF) passou a disponibilizar dados
relacionados às ocorrências dos acidentes em rodovias federais. Todavia, são apenas
dados disponibilizados em planilhas. Desse modo, não sendo possível fazer interpretações
desses dados de forma fácil, eficiente e eficaz.

A PRF possui um software que faz representações visuais estatísticas, todavia é
destinado a uso interno e, atualmente (Abril, 2015), está fora do ar . Com o intuito de
transformar esses dados em informações visualmente apresentáveis para os cidadãos, a
aplicação “Pé na Estrada” utiliza-os para exibir informações estatísticas acerca dos
acidentes, com a finalidade de diminuir a ocorrência dos acidentes em rodovias federais.
Contudo, o modelo de dados atual está mal estruturado e apresenta inconsistências e
informações incompletas, o que diminui a representatividade estatística do software e
também reduz a capacidade do software de mostrar valor ao usuário, limitando suas
funcionalidades a somente mostrar os acidentes numa rota, disponibilizar um ranking das
rodovias em relação aos acidentes ocorridos e fazer comentários sobre uma rodovia.