\chapter{\textit{Software} 'Pé na Estrada'}

A disponibilização das ocorrências de acidentes em rodovias federais através da PRF,
embora exista, não é realizada de forma visual. Todavia, consiste na disponibilização de
planilhas complexas que contém muitos dados.

O software “Pé na estrada” foi desenvolvido, utilizando os dados divulgados pela
PRF, com a finalidade de apresentar essas informações de forma mais atrativa e visual. Dessa
forma, é um sistema que permite buscar informações sobre as rodovias federais, visualizando
as rodovias com maiores índices de acidentes. A aplicação também permite traçar rotas para
visualizar os trechos mais perigosos, e os acidentes ocorridos ao longo da rota. Há também a
possibilidade dos usuários fazerem comentários sobre as rodovias. Algumas imagens do
software podem ser vistas no Anexo A.

Os dados disponibilizados possuem muitas informações acerca do acidente, como tipo
do acidente, dano causado à rodovia e tipo de veículo. Contudo, esse tipo de informação não
foi tratado na aplicação, pois esses dados estavam estruturados de forma complexa e
inconsistente. Os relacionamentos entre os dados eram feitos por meio de uma grande
quantidade de tabelas relacionadas, o que dificultaria a apresentação desses dados na
aplicação, tornando o sistema lento e demasiado complexo de se manter. O modelo de dados
utilizado na aplicação pode ser visto na Figura \ref{fig:modelo_penaestrada}.

\begin{figure}[!htb]
 \centering
 \includegraphics[scale = 0.4]{modelo_penaestrada}
 \caption[Modelo de dados do software “Pé na Estrada”]{Modelo de dados do software “Pé na Estrada”.}
 \label{fig:modelo_penaestrada}
\end{figure}

  \section{Mudanças no \textit{Software}}
  
    Atualmente, o “Pé na estrada” disponibiliza informações apenas sobre os locais dos
    acidentes, e a quantidade de acidentes em cada rodovia. Esse tipo de informação seria mais
    relevante se acompanhassem informações acerca dos tipos de acidentes, danos causados a
    rodovia, tipo de veículo, envolvidos, entre outros dados que fossem relevantes para análise
    estatística e para representação gráfica.

    Dessa forma, com o modelo de dados estruturado semanticamente, esses dados seriam
    encontrados e relacionados facilmente, possibilitando ao software disponibilizar essas
    informações ao usuário, até mesmo de forma mais visual.

    Atualmente, o software não possui nenhuma camada semântica presente, possuindo
    apenas a camada visual do HTML, sendo necessária toda a estruturação das três principais
    camadas semânticas: o XML, o RDFS e a OWL, sendo esta última necessária para o uso da
    ontologia que será criada para representar os acidentes de trânsito.

    Mudanças na estrutura do banco de dados utilizado também serão necessárias para
    comportar os novos tipos de dados que irão surgir. Como o software não possui as
    funcionalidades inerentes a outros tipos de dados sobre acidentes (como tipo do acidente, tipo
    de veículo, envolvidos, etc.), estas terão que ser criadas, impactando significativamente na
    estrutura do código e, talvez, da aplicação em si.
    
    