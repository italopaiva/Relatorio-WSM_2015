\chapter{Construção da ontologia}
  
  Este capítulo apresenta detalhes da construção da ontologia proposta.

  \section{Metodologia utilizada para construção}
    
      A partir da análise conceitual dos métodos e metodologias para construção de ontologias abordados neste trabalho, 
      o método escolhido para a realização deste projeto, levando em conta as etapas/fases propostas por cada uma dessas 
      metodologias ou métodos no que tange ao processo de construção de ontologias, foi o método \textit{\textbf{Ontology 101}}.
  
      Os aspectos observados na escolha do método foram: \textbf{a)} suporte literário bem consolidado e disponível 
      (livros, artigos, guias e etc); \textbf{b)} o método é baseado na literatura do paradigma orientado a objetos;
      \textbf{c)} familiarização por parte dos integrantes do grupo com os conceitos e processos propostos pelo método.
    
  \section{Escopo da ontologia}
      
      A ontologia proposta neste trabalho será utilizada para o contexto de acidentes de veículos ocorridos em rodovias
      federais brasileiras, portanto, alguns conceitos inerentes a outros tipos de acidentes não se aplicam a este 
      contexto. Por esse motivo, os modelos conceituais das ontologias encontradas foram analisados e utilizados como base 
      para a construção da ontologia final.
      
      Para este trabalho, é esperado apenas uma ontologia capaz de oferecer um arcabouço mais semântico para o \textit{software} 
      "Pé na Estrada", de modo a provocar mudanças iniciais significativas no \textit{software}.
      Para trabalhos futuros, pretende-se utilizar ontologias específicas para as entidades como Pessoa, Veículo, Rodovia e
      Hospital para aumentar a representatividade da ontologia, 
      além de incluir atributos mais específicos para cada classe.
  
  \vfill
  \pagebreak
  \section{Ontologias encontradas}
   
      \chapter{Ontologias de acidentes}

  A fim de não ter retrabalho desnecessário, e de usar os conhecimentos previamente
  pesquisados por outros grupos de pesquisa, este grupo procurou em bases de periódicos e no
  próprio Google por ontologias que já existissem, e pudessem retratar de forma semântica os
  acidentes listados no software “Pé na Estrada”. As consultas muitas vezes não retornaram
  resultados satisfatórios, mas foi possível encontrar duas ontologias muito parecidas.
  
  De posse das ontologias, é possível observar a similaridade entre as duas encontradas,
  até por que as duas são empregadas em soluções tecnológicas parecidas. Sua aplicação
  consiste na organização das informações de acidentes em estradas, para que as informações de
  um acidente de trânsito sejam devidamente transmitidas em redes veiculares Ad hoc, ou
  VANETs \cite{barrachina12}. Essas redes usam os carros como nós de uma rede para
  trafegar informações, que podem ser analisados e registrados em banco de dados. Na
  aplicação dos projetos encontrados ela também tem a utilidade de alertar ambulâncias, ou
  hospitais próximos que possam prestar socorro de forma rápida e eficiente. As duas
  ontologias possuem as mesmas entidades e seus relacionamentos são idênticos, evidenciando
  a robustez das mesmas. A diferença está nos atributos das entidades, com a adição de diversos
  campos que fornecem informações valiosas para a correta identificação do acidente de
  trânsito \cite{villalba14}.
  
  Desta forma, as duas ontologias são tomadas como base para o contexto deste projeto,
  pois se adequam muito bem dentro dos objetivos almejados. Será proposta a adição de uma
  entidade, de forma que seja possível também captar informações sobre a rodovia a qual o
  acidente ocorreu, diferentemente da entidade ambiente (“environment”), que procura detalhar
  as condições no momento do acidente. A entidade rodovia terá atributos que irão guardar as
  informações daquela rodovia, relacionadas ao contexto do projeto. O detalhamento dessa
  entidade será conduzido no trabalho seguinte.
  
  As duas ontologias podem ser vistas na Figura \ref{fig:ontologia1} e na Figura \ref{fig:ontologia2}.
  
  \begin{figure}[!htb]
    \centering
    \includegraphics[scale = 0.6]{ontologia1}
    \caption[Componentes da ontologia]{Componentes da ontologia. Fonte: \cite{villalba14}}
    \label{fig:ontologia1}
  \end{figure}
  
    \begin{figure}[!htb]
    \centering
    \includegraphics[scale = 0.6]{ontologia2}
    \caption[Componentes da ontologia CAOVA]{Componentes da ontologia CAOVA. Fonte: \cite{barrachina12}}
    \label{fig:ontologia2}
  \end{figure}
      
  \vfill
  \pagebreak
  \section{Estrutura da ontologia}

      Com a escolha da metodologia 101 a primeira etapa consiste na definição de classes na ontologia.
      Para essa definição foi realizada uma técnica chamada \textit{CardSorting}. A Figura \ref{fig:card_sorting} 
      ilustra os conceitos levantados pela equipe com a técnica \textit{CardSorting}.
      
      \begin{figure}[!htb]
	\centering
	\includegraphics[scale = 0.1]{card_sorting}
	\caption[Conceitos levantados com o \textit{CardSorting}]{Conceitos levantados com o \textit{CardSorting}}
	\label{fig:card_sorting}
      \end{figure}
      
      Essa técnica consiste na escrita de termos em cartões que representem a interação do usuário com a aplicação,
      quais seriam as perguntas que ele faria ao sistema, o que ele procuraria. Dessa forma, pensando como usuário
      da aplicação foi realizado um levantamento de termos.

      A partir do levantamento dos principais termos, foi realizado um mapa conceitual inicial com o intuito
      de organizar as classes em uma hierarquia e para identificar seus relacionamentos, que consiste na segunda
      etapa da construção.

      \subsection{Classes e Propriedades}
      
	  Como não foi possível encontrar o código OWL das ontologias pesquisadas, a equipe decidiu aproveitar a
	  modelagem conceitual feita nos trabalhos encontrados. A partir desses modelos, foi construído o modelo 
	  para a ontologia, ilustrado na Figura \ref{fig:modelo_conceitual_ontologia}.
	  
	  Após a elaboração do mapa conceitual inicial as propriedades entre as classes foram definidas e as ontologias
	  encontradas foram utilizadas. 
	  Na tabela \ref{tab:classes} está a lista de classes e as propriedades que elas possuem.      

	  \begin{table*}[!h]
	  \centering
	  \begin{tabular}{p{0.2\linewidth}p{0.20\linewidth}p{0.20\linewidth}}
	    \hline
	    \textbf{Classe} & \textbf{Propriedade} & \textbf{Objeto (Classe)}\\
	    \hline
	      Acidente & tem\_tipo & TipoAcidente\\
		& tem\_causa & Causa\\
		& ocorre & Local\\
		& envolve & Veiculo\\
	    \hline
	      Veículo & esta\_envolvido & Acidente\\
		& tem\_ocupante & Ocupante\\
	    \hline
	      Ocupante & viaja\_em & Veiculo\\
		& motorista & Veiculo\\
		& proprietario & Veiculo\\
		& atendido\_em & Hospital\\
	    \hline
	      Local & tem\_acidente & Acidente\\
		& esta\_contido & Rodovia\\
	    \hline
	      Rodovia & tem\_estatistica & Estatistica\\
		& contem & Local\\
		& tem\_postos\_prf & PostosPRF\\
		& tem\_hospital & Hospital\\
		& tem\_assistencia & AssistenciaTecnica\\
	    \hline
	      Estatística & acerca & Rodovia\\
	    \hline
	      Causa & - & -\\
	    \hline
	      TipoAcidente & - & -\\
	    \hline
	      PostoPRF & - & -\\
	    \hline
	      AssistenciaTecnica & - & -\\
	    \hline
	      Hospital & - & -\\
	    \hline
	  \end{tabular}
	  \caption{Classes e Propriedades}
	  \label{tab:classes}
	  \end{table*}
	  
	  Os relacionamentos entre as classes podem ser vistos melhor no modelo conceitual da ontologia ilustrado na
	  Figura \ref{fig:modelo_conceitual_ontologia}.
	  
	  \vfill
	  \pagebreak
	  \begin{figure}[!h]
	    \centering
	    \includegraphics[scale = 0.4]{modelo_conceitual_ontologia.png}
	    \caption[Modelo conceitual ideal da ontologia]{Modelo conceitual ideal da ontologia a ser construída}
	    \label{fig:modelo_conceitual_ontologia}
	  \end{figure}
	  
	  O modelo conceitual apresentado se assemelha bastante com o padrão do grafo final que se obteria com a construção da
	  ontologia.
      
      \vfill
      \pagebreak
      \subsection{Atributos das classes}
	
	  Apesar de não estar descrito no modelo conceitual apresentado, por questão de organização, as classes levantadas
	  possuem atributos importantes que serão descritos nesse tópico.
	  
	  Para a definição dos atributos das classes foram considerados alguns atributos definidos no modelo de dados atual
	  da PRF.
	  
	  % Classe Acidente
\noindent
\emph{\textbf{Acidente}}

  A classe Acidente pode ser considerada como a classe central para a aplicação e para a ontologia,
  uma vez que todas as outras classes estão relacionadas a um acidente ocorrido.
  
    \begin{table*}[!h]
    \centering
    \begin{tabular}{p{0.15\linewidth}p{0.2\linewidth}p{0.5\linewidth}}
      \hline
      \textbf{Classe} & \textbf{Atributo} & \textbf{Descrição}\\
      \hline
	Acidente & data\_acidente & Data da ocorrência do acidente\\
		  & hora\_acidente & Horário da ocorrência do acidente\\ 
      \hline
    \end{tabular}
    \caption{Atributos da classe Acidente}
    \label{tab:attr_acidente}
    \end{table*}
    
% Classe Veículo
\noindent
\emph{\textbf{Veículo}}

  A classe Veículo representa os veículos envolvidos em um acidente.
  
    \begin{table*}[!h]
    \centering
    \begin{tabular}{p{0.15\linewidth}p{0.22\linewidth}p{0.5\linewidth}}
      \hline
      \textbf{Classe} & \textbf{Atributo} & \textbf{Descrição}\\
      \hline
	Veiculo & tipo\_veiculo & Tipo do veículo (Ex.: Hatch, sedan, caminhonete)\\
		& modelo\_veiculo & Modelo do veículo\\
		& ano\_veiculo & Ano do veículo\\		
		& km\_veiculo & Quilometragem do veículo no momento do acidente\\
		& defeitos\_veiculo & Defeitos encontrados no veículo no momento do acidente\\
		& quantidade\_pessoas & Quantidade de pessoas dentro do veículo no momento do acidente\\
      \hline
    \end{tabular}
    \caption{Atributos da classe Veículo}
    \label{tab:attr_veiculo}
    \end{table*}

\vfill
\pagebreak

% Classe Ocupante
\noindent
\emph{\textbf{Ocupante}}

  A classe Ocupante representa as pessoas que podem estar dentro de um veículo no momento do acidente.
  
    \begin{table*}[!h]
    \centering
    \begin{tabular}{p{0.15\linewidth}p{0.23\linewidth}p{0.5\linewidth}}
      \hline
      \textbf{Classe} & \textbf{Atributo} & \textbf{Descrição}\\
      \hline
	Ocupante & sexo & Modelo do veículo\\
		 & idade\_pessoa & Idade da pessoa\\
		 & usava\_cinto & Indica se o ocupante do veículo estava usando o cinto de segurança\\
		 & estado\_mental* & Indica o estado mental do motorista no momento do acidente\\
		 & sob\_efeito\_toxicos* & Indica se o motorista estava sob efeito de tóxicos no momento do acidente\\
		 & uf\_cnh\_motorista* & UF da CNH do motorista\\
      \hline
    \end{tabular}
    *Atributos específicos para instâncias com a relação 'motorista'.

    \caption{Atributos da classe Ocupante}
    \label{tab:attr_ocupante}
    \end{table*}
    
% Classe Local
\noindent
\emph{\textbf{Local}}

  A classe Local representa o local, dentro de uma rodovia, no qual um acidente ocorreu.
  
    \begin{table*}[!h]
    \centering
    \begin{tabular}{p{0.15\linewidth}p{0.23\linewidth}p{0.5\linewidth}}
      \hline
      \textbf{Classe} & \textbf{Atributo} & \textbf{Descrição}\\
      \hline
	Local & latitude & Latitude em que ocorreu o acidente\\
	      & longitude & Longitude em que ocorreu o acidente\\
	      & km\_rodovia & Indica em qual quilômetro da rodovia ocorreu o acidente\\
	      & ponto\_referencia & Indica um ponto de referência próximo ao local do acidente\\
	      & regiao\_urbana & Indica se a área do acidente é uma região urbana\\
	      & volume\_trafego & Indica o volume de tráfego na região\\
      \hline
    \end{tabular}
    \caption{Atributos da classe Local}
    \label{tab:attr_local}
    \end{table*}

\vfill
\pagebreak
    
% Classe Rodovia
\noindent
\emph{\textbf{Rodovia}}

  A classe Rodovia representa as rodovias federais brasileiras. No domínio da aplicação, todo acidente ocorre em uma rodovia.
  
    \begin{table*}[!h]
    \centering
    \begin{tabular}{p{0.15\linewidth}p{0.23\linewidth}p{0.5\linewidth}}
      \hline
      \textbf{Classe} & \textbf{Atributo} & \textbf{Descrição}\\
      \hline
	Rodovia & nome\_rodovia & Nome da rodovia (Ex.: BR-060)\\
		& extensao\_rodovia & Indica o tamanho da rodovia\\
		& estado\_rodovia & Identifica quais estados brasileiros que a rodovia percorre\\
		& posto\_prf & Identifica um posto da PRF ao longo da rodovia\\
      \hline
    \end{tabular}
    \caption{Atributos da classe Rodovia}
    \label{tab:attr_rodovia}
    \end{table*}
    
\vfill    
      
%       \pagebreak
%       \subsection{Modelagem da ontologia no Protégé}
%       
% 	As classes e propriedades definidas foram modeladas na ferramenta Protégé \footnotemark[2],
% 	de acordo com o modelo conceitual final	apresentado na Figura \ref{fig:modelo_conceitual_ontologia}.
% 	\footnotetext[2]{http://protege.stanford.edu/}
% 	
% 	As Figuras \ref{fig:classes_protege} e \ref{fig:propriedades_protege} apresentam, respectivamente, as classes e
% 	as propriedades que foram modeladas no Protégé.
% 	
% 	\begin{figure}[!htb]
% 	  \centering
% 	  \includegraphics[scale = 0.55]{classes_wsm}
% 	  \caption[Classes modeladas no Protégé]{Classes modeladas no Protégé.}
% 	  \label{fig:classes_protege}
% 	\end{figure}
% 	
% 	\begin{figure}[!htb]
% 	  \centering
% 	  \includegraphics[scale = 0.6]{propriedades_classes_wsm}
% 	  \caption[Propriedades das classes modeladas no Protégé]{Propriedades das classes modeladas no Protégé.}
% 	  \label{fig:propriedades_protege}
% 	\end{figure}
% 	
%       \pagebreak
%       \section{Cronograma para Construção da Ontologia na Aplicação}

A aplicação da ontologia no software envolvem algumas mudanças estruturais na arquitetura e
na refatoração de algumas funcionalidades, como uma busca no banco de dados por um acidente por 
exemplo. Desta forma, a equipe técnica definiu a construção da ontologia na aplicação como um projeto
separado, com cronograma próprio, integrado com o cronograma geral do projeto.
O objetivo é encarar a construção como uma mudança de requisitos, estudando o impacto e analisando
as implicações do uso da ontologia dentro da ferramenta.

\begin{figure}[h]
	\centering
	\includegraphics{Figuras/cronograma_construcao_ontologia.png}
	content...
\end{figure}

\pagebreak

Tabela – Cronograma de Construção da Ontologia na Aplicação
