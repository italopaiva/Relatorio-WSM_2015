\chapter{Construção da ontologia}

  \section{Metodologia utilizada para construção}
    
    APRESENTAR AQUI A METODOLOGIA ESCOLHIDA PARA A CONSTRUÇÃO DA ONTOLOGIA (Ontology 101).
    
  \section{Construção da ontologia}
    
    \subsection{Escopo da ontologia}
    
    \subsection{Ontologias encontradas}
   
      \chapter{Ontologias de acidentes}

  A fim de não ter retrabalho desnecessário, e de usar os conhecimentos previamente
  pesquisados por outros grupos de pesquisa, este grupo procurou em bases de periódicos e no
  próprio Google por ontologias que já existissem, e pudessem retratar de forma semântica os
  acidentes listados no software “Pé na Estrada”. As consultas muitas vezes não retornaram
  resultados satisfatórios, mas foi possível encontrar duas ontologias muito parecidas.
  
  De posse das ontologias, é possível observar a similaridade entre as duas encontradas,
  até por que as duas são empregadas em soluções tecnológicas parecidas. Sua aplicação
  consiste na organização das informações de acidentes em estradas, para que as informações de
  um acidente de trânsito sejam devidamente transmitidas em redes veiculares Ad hoc, ou
  VANETs \cite{barrachina12}. Essas redes usam os carros como nós de uma rede para
  trafegar informações, que podem ser analisados e registrados em banco de dados. Na
  aplicação dos projetos encontrados ela também tem a utilidade de alertar ambulâncias, ou
  hospitais próximos que possam prestar socorro de forma rápida e eficiente. As duas
  ontologias possuem as mesmas entidades e seus relacionamentos são idênticos, evidenciando
  a robustez das mesmas. A diferença está nos atributos das entidades, com a adição de diversos
  campos que fornecem informações valiosas para a correta identificação do acidente de
  trânsito \cite{villalba14}.
  
  Desta forma, as duas ontologias são tomadas como base para o contexto deste projeto,
  pois se adequam muito bem dentro dos objetivos almejados. Será proposta a adição de uma
  entidade, de forma que seja possível também captar informações sobre a rodovia a qual o
  acidente ocorreu, diferentemente da entidade ambiente (“environment”), que procura detalhar
  as condições no momento do acidente. A entidade rodovia terá atributos que irão guardar as
  informações daquela rodovia, relacionadas ao contexto do projeto. O detalhamento dessa
  entidade será conduzido no trabalho seguinte.
  
  As duas ontologias podem ser vistas na Figura \ref{fig:ontologia1} e na Figura \ref{fig:ontologia2}.
  
  \begin{figure}[!htb]
    \centering
    \includegraphics[scale = 0.6]{ontologia1}
    \caption[Componentes da ontologia]{Componentes da ontologia. Fonte: \cite{villalba14}}
    \label{fig:ontologia1}
  \end{figure}
  
    \begin{figure}[!htb]
    \centering
    \includegraphics[scale = 0.6]{ontologia2}
    \caption[Componentes da ontologia CAOVA]{Componentes da ontologia CAOVA. Fonte: \cite{barrachina12}}
    \label{fig:ontologia2}
  \end{figure}
      
    \pagebreak
    \subsection{Estrutura da ontologia}

	  Com a escolha da metodologia 101 a primeira etapa consiste na definição de classes na ontologia.
	  Para essa definição foi realizada uma técnica chamada \textit{CardSorting}. 
	  
	  Essa técnica consiste na escrita de termos em cartões que representem a interação do usuário com a aplicação,
	  quais seriam as perguntas que ele faria ao sistema, o que ele procuraria. Dessa forma, pensando como usuário
	  da aplicação foi realizado um levantamento de termos.

	  A partir do levantamento dos principais termos, foi realizado um mapa conceitual inicial com o intuito
	  de organizar as classes em uma hierarquia, que consiste na segunda etapa da construção.

      \subsubsection{Classes e Propriedades}
      
      Como não foi possível encontrar o código OWL das ontologias pesquisadas, a equipe decidiu aproveitar a
      modelagem conceitual feita nos trabalhos encontrados. A partir desses modelos, foi construído o modelo 
      para a ontologia, ilustrado na figura \ref{fig:modelo_conceitual_ontologia}.
      
      Após a elaboração do mapa conceitual inicial as propriedades entre as classes foram definidas e as ontologias encontradas foram utilizadas. 
      Na tabela \ref{tab:classes} está a lista de classes
      e as propriedades que elas possuem.      

      \begin{table*}[!h]
      \centering
      \begin{tabular}{p{0.15\linewidth}p{0.15\linewidth}}
	\hline
	\textbf{Classe} & \textbf{Propriedade} \\
	\hline
	  Acidente & tem\_tipo\\
	    & tem\_causa\\
	    & ocorre\\
	    & esta\_contido\\
	\hline
	  Veículo & esta\_envolvido\\
	    & tem\_ocupante\\
	\hline
	  Ocupante & viaja\_em\\
	    & motorista\\
	    & proprietario\\
	\hline
	  Local & tem\_acidente\\
	    & tem\_coordenada\\
	    & esta\_contido\\
	    & condicoes\_rodovias\\
	\hline
	  Rodovia & tem\_estatistica\\
	    & contem\\
	    & tem\_postos\_prf\\
	    & tem\_hospital\\
	    & tem\_assistencia\\
	\hline
	  Estatística & acerca\\
	\hline
      \end{tabular}
      \caption{Classes e Propriedades}
      \label{tab:classes}
      \end{table*}
      
      Os relacionamentos entre as classes podem ser vistos melhor no modelo conceitual da ontologia ilustrado na
      Figura \ref{fig:modelo_conceitual_ontologia}.\\
      
      \begin{figure}[!htb]
	\centering
	\includegraphics[scale = 0.23]{modelo_conceitual_ontologia}
	\caption[Modelo conceitual da ontologia]{Modelo conceitual da ontologia a ser construída}
	\label{fig:modelo_conceitual_ontologia}
      \end{figure}
      
      Essas classes foram modeladas na ferramenta Protégé \footnotemark[2], de acordo com o modelo conceitual final
      apresentado na Figura \ref{fig:modelo_conceitual_ontologia}.

      \footnotetext[2]{http://protege.stanford.edu/}
	

      
      
    
    
      
  

  