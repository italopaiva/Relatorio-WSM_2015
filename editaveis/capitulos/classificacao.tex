\section{Classificações de Terminologias}

As ontologias podem ser classificadas em diversos aspectos, sobre o seu espectro
semântico, quanto a sua generalidade e quanto ao tipo de informação que representam.
Elas foram propostas por alguns estudiosos da área.

As ontologias são classificadas quanto a generalidade, tem a generalidade da ontologia
como principal critério de classificação. Foram identificadas por Nicola Guarino do
seguinte modo:
\begin{itemize}
  \item Ontologias de nível superior;
  \item Ontologias de domínio;
  \item Ontologias de tarefas;
  \item Ontologias de aplicação.
\end{itemize}
As ontologias quanto ao tipo de informação que representam foram propostas por
Assunción Gomez Perez, Mariano Fernandez Lopez e Oscar Corcho. Nelas o tipo de
informação a ser modelado é muito importante \cite{breitman05}. E foram classificadas em:

\begin{itemize}
\item Ontologias para a representação do conhecimento;
\item Ontologias gerais e de uso comum;
\item Ontologias de topo e de nível superior (upper ontologies);
\item Ontologias de domínio;
\item Ontologias de tarefas;
\item Ontologias de domínio-tarefa;
\item Ontologias de métodos;
\item Ontologias de aplicação.
\end{itemize}

O principal tipo de classificação de ontologias (pelo menos se tratando dentro do
âmbito da web semântica) é a classificação para ontologias segundo o seu espectro
semântico, que foram propostas na literatura por Lassila e Deborah McGuiness. Esse tipo
de classificação difere-se desde a mais “leve” (lightweight) até a mais “pesada”
(heavyweight). E essa forma, segundo o seu espectro semântico, classifica-se baseada na
estrutura interna e no conteúdo das ontologias \cite{breitman05} . Foram divididas em:

\begin{itemize}
\item Vocabulários controlados;
\item Glossários;
\item Tesauros;
\item Hierarquias tipo-de-informais
\item Hierarquia tipo-de-formais;
\item Frames;
\item Ontologias que exprimem restrições de valores;
\item Ontologias que exprimem restrições lógicas.
\end{itemize}

Os vocabulários controlados consistem em uma lista de termos referentes à aplicação.
Glossário é onde está exposta uma lista de termos, assim como os vocabulários controlados,
só que em linguagem natural. Já os tesauros são de extrema importância, pois um tesauro
reúne um conjunto de relacionamentos entre termos que estão organizados dentro de uma
taxonomia.

Taxonomia é definida pelo dicionário Merriam Webster (\textit{apud} \citeauthor{breitman05}, \citeyear{breitman05})
como: “O estudo dos princípios gerais da classificação científica: classificaçao sistemática,11
em particular, classificação ordenada de plantas e animais segundo relacionamentos naturais”.
Assim, é a classificação de entidades de informação segundo a hierarquia que representam.
Com esses conceitos em mente é possível definir tesauros como uma taxonomia adicionada a
um conjunto de relacionamentos semântico entre seus termos. Tesauros são consistentes, pois
podem servir como indexadores a várias bases de dados através de uma terminologia
\cite{breitman05}.

Já as Hierarquias de tipos (formais e informais) têm a sua relevância. Na primeira, os
relacionamentos de generalização são respeitados integralmente. E nas de tipo informal, os
relacionamentos de generalização são realizados de maneira informal, sendo essa a diferença
principal.

Os frames são compostos por classes e propriedades. As primitivas de um modelo de
frames são classes que possuem propriedades chamadas atributos ou slots. Cada frame oferece
o contexto para modelar um aspecto de domínio. Já os atributos, ou slots, se aplicam somente
na classe a qual foram definidas, e não em um aspecto global \cite{breitman05}.

Já as ontologias que exprimem restrições lógicas permitem expressar restrições em
lógica de primeira ordem. E em relação às ontologias que exprimem restrições de valores elas
trabalham oferecendo subsídio para restringir os valores assumidos pelas propriedades da
classe em questão \cite{breitman05}.

Para este trabalho, será utilizado um Glossário de termos técnicos rodoviários, criado
pelo Departamento Nacional de Estradas de Rodagem \cite{brasil97}.

